The main objectives of this work were to explore the children's drawings dataset to provide preliminary insights and to develop a system that retrieves a famous work (painting, sculpture, poster, portrait, or photo of a landmark/building) that is similar to a child's drawing.

Drawings datasets are usually associated with sketch datasets \cite{eitz2012hdhso, jongejan2016quick, Patsorn2016SketchyDatabase, Konyushkova2015GodsKD}, which are useful for tasks of image retrieval, semantic clustering using sketches, generating sketches using deep neural networks or in education, investigating psychology or crime. However, these datasets suffer some drawbacks. First, most of the datasets contain only grayscale pencil sketches. Second, the artist's age is available only in one or two datasets. Due to these deficiencies, along with not being child-centric and focusing on a singular object, they lack the diversity contained in the digitized IMAJ-UNESCO children's drawings dataset. The children's drawings dataset has drawings created by artists aged between 3 and 25 from 1994 till today based on a distinctive theme each year, making it a spatially, temporally, and conceptually diverse dataset. 

The drawings have deeply rooted cultural references that vary geographically. Although children sometimes use a different drawing technique or reflect on global issues, their drawings contain elements of local culture. Few drawings also indicate that children do not hesitate to discuss uncomfortable or sensitive topics in their drawings. Many drawings have their references implanted in modern-day cultural objects. An important observation that stays valid across all the drawings in the datasets is that children tend to reuse a style/technique of a famous work than completely replicating it.

Digitizing and producing digital copies of the physical works is only one side. On the other side, these novel datasets pose challenges regarding storage (indexing), access, and search and retrieval. The exploration of the dataset identified some challenges specific to it. The partly solved first challenge is the correctness of the current metadata. The information in the file name does not always corroborate with the drawings' information. This discrepancy can lead to incorrect conclusions about the creator's country or age and the interpretation of the work. The second challenge lies in extracting the metadata. At the current state of digitization, it is impossible to attribute the drawing to an artist. While most drawings contain the name, OCRizing non-uniform handwritten text is exigent.

The experiments using various CNN models explored the feasibility of retrieving artworks similar to drawings. The findings show that the CNN-based deep image retrieval techniques can efficiently solve the problem of matching drawings and artworks that share a visual similarity. Experimentation involved the CNN models trained for image classification (Baseline), pattern matching (Mini-Replica), and pattern clustering (Clus-Replica) in paintings. The models trained for pattern recognition in paintings achieve good performance than the model trained for classification on ImageNet. Fine-tuning these models enhance their retrieval capability. The best model is the fine-tuned version of a model trained to retrieve artwork photos that contain a similar visual pattern. It achieves an average recall of 88\% at a threshold of the top 400 results, almost double that of the model trained for image classification and more than 12 percentage points compared to its non-fine-tuned version. The recall and mean average precision measures double in the best model compared to the baseline image classification model. Fine-tuning the baseline model with relevant examples of drawings and artworks gives metrics similar to the best model.

The style augmentation process that adds watercolor, oil painting, pencil sketch, and texture has a consequential effect on the model's performance. It boosts the evaluation metrics and, most importantly, increases the models' generalization ability. This style transfer helpfully communicates the visual similarity between the drawings and artworks in an otherwise disparate set of images and guides the CNN model in learning those differentiating features.

Qualitative evaluation of the models provides insights into their routine. The baseline model results are as good as the other models only when the drawing is an identical duplicate of the artwork. However, it retrieves images that use the same technique as the drawing. While the Replica variant models get over this, as often as not, the top-ranked artworks by the Clus-Replica model are semantically different from the drawings. The Mini-Replica and the task-specific fine-tuned baseline models learn the object localization and retrieve visually and semantically similar artworks.

Finally, analogizing the comparable performance of the fine-tuned baseline model with Mini-Replica and its fine-tuned version with the size of the training data uncloaks the need for quality data. It shows that even in small quantities, meticulously curated data helps create a deep learning system that performs up to the mark. This thought broadly falls into the realm of Data-Centric AI. As introduced by Andrew Ng\footnote{\url{https://landing.ai/data-centric-ai/}}, Data-Centric AI shifts the focus from improving the code (model) to improving systems for creating efficient and high-quality datasets to improve the overall system's performance.

In conclusion, this work opens a new direction of work in exploring children's drawings, different from using those drawings for psychological or developmental analysis. The project demonstrates the possibility of identifying visually similar images by crossing the barriers of domains, techniques, and creators' age and hopefully inspires exploring the children's drawings through an artistic lens and using computer vision techniques for such tasks. The works searching for similarities in style, subjects, and objects of the drawings and famous cultural objects can come together and create a tool to extract artistic inspirations from a drawing.